\begin{enumerate}
    \begin{comment}
%% Begin Exercise
\item 
Below are listed several simplex tableaux.  Each represents a maximization LP at
a basic feasible solution.  

Your goal is to analyze the tableaux thoroughly. 
Some suggestions are the following:
\begin{enumerate}
\item Determine which variables are basic and which are non-basic.
\item List the current basic feasible solution (including all variables and their values).
\item 
\begin{enumerate}
\item  Perform an appropriate simplex pivot, explaining why you
  chose the entering and leaving variables as you did, or 
\item  Explain why no pivot is possible/appropriate and what that says about the LP.
\end{enumerate}
\item If you can, determine whether the LP, for example,
is unbounded, is infeasible, is degenerate, is at optimum, has alternate optima,
\emph{etc}.
\end{enumerate}

\begin{center}
  \begin{tabular}{c}
  Tableau I\\
  \simpmat{3}{3}{
    z &x_{1} & x_{2} &x_{3} & s_{1}& s_{2} &s_{3} &\rhs \\ \hline
    1 & 0    & -2    & 0    &  1   & 0     & 0    & 10  \\
    0 & 0    & -1    & 0    &  1   & 1     & 5    &  2  \\
    0 & 1    & -4    & 0    &  4   & 0     & 8    &  6  \\
    0 & 0    &  0    & 1    &  4   & 0     &-1    &  8  }
\end{tabular}
  \hfill
  \begin{tabular}{c}
  Tableau II \\
  \simpmat{3}{3}{
    z &x_{1} & x_{2} &x_{3} & s_{1}& s_{2} &s_{3} &\rhs \\ \hline
    1 &-2    &  0    & 1    &  0   &  2    & 0    & 6   \\ 
    0 & 2    &  1    & 6    &  0   &  5    & 0    & 5   \\
    0 & 4    &  0    & 8    &  1   & 12    & 0    & 8   \\
    0 & 3    &  0    & 9    &  0   & -2    & 1    & 9   }
\end{tabular}
  \\[1.5ex]
  \begin{tabular}{c} 
    Tableau III\\
  \simpmat{3}{3}{
    z &x_{1} & x_{2} &x_{3} & s_{1}& s_{2} &s_{3} &\rhs \\ \hline
    1 & 0    &  1    & 0    & 4    &  0    & 0    & 8  \\
    0 & 0    &  0    & 1    & 1    &  1    & 0    & 3  \\
    0 & 0    & -1    &-3    & 1    &  0    & 1    & 4  \\
    0 & 1    &  1    & 1    &-2    &  0    & 0    & 5
  }
\end{tabular}
\end{center}
%% End Exercise

\clearpage
\end{comment}
%% Begin Exercise
\item Consider the following Linear Program (LP):
\begin{equation*}
  \LP{3}{
\text{maximize\quad}	 &z&=& 2x_1 &   &      & + & 9x_3 &      &    \\\notag
\text{subject to:\quad}  & & & 3x_1 & + & 2x_2 & - &  x_3 & \geq & 10 \\\notag
                     	 & & &  x_1 & - &  x_2 &   &      & \leq &  0 \\\notag
                     	 & & &  x_1 & + & 3x_2 & + &  x_3 & \leq & 11 \\\notag
			 & & &\multicolumn{7}{c}{x_1, x_2, x_3 \geq 0}
		       }
\end{equation*}
\begin{enumerate}
  \item Put the LP into standard inequality (\(\le\)) form.

  \item For each of the following solutions, determine whether the solution is feasible or not, and
    explain why.
    \begin{multicols}{2}
      \begin{enumerate}
	\item 	\(x_{1}=0,x_{2}=0,x_{3}=0\)
	\item 	\(x_{1}=0,x_{2}=0,x_{3}=11\)
	\item 	\(x_{1}=0,x_{2}=0,x_{3}=-10\)
	\item 	\(x_{1}=2,x_{2}=2,x_{3}=0\)
      \end{enumerate}
    \end{multicols}
  \item For each of the following solutions, evaluate the objective function value at the solution.
    \begin{multicols}{2}
      \begin{enumerate}
	\item 	\(x_{1}=0,x_{2}=0,x_{3}=0\)
	\item 	\(x_{1}=0,x_{2}=0,x_{3}=11\)
	\item 	\(x_{1}=0,x_{2}=0,x_{3}=-10\)
	\item 	\(x_{1}=2,x_{2}=2,x_{3}=0\)
      \end{enumerate}
    \end{multicols}
\end{enumerate}
%% End Exercise

\clearpage
%% Begin Exercise
\item 
Consider the following LP (where the objective function is unspecified):
\begin{equation*}\LP{2}{
    \text{maximize}&z&=&\multicolumn{3}{c}{f(x_{1},x_{2})}\\
    \text{subject to}& & & -x_1 & + &  x_2 & \leq & 2\\\notag
                     & & &  x_1 & - & 2x_2 & \leq & 1\\\notag
	             & & & \multicolumn{5}{c}{x_1,x_2\geq0}
	 }
\end{equation*}
\begin{enumerate}
\item Sketch and shade the feasible region. You may use the grid in \tabref[vref]{sketchFeasible}.
\item Is the \emph{feasible region} bounded or unbounded? How can you tell?
\item Assuming that the objective for the LP is
\begin{equation*}
  \LP{2}{\text{maximize}&z&=&x_{1}&-&3x_{2}}
\end{equation*}
\begin{enumerate}
  \item Draw and label (with the objective value) several contour lines for this objective function;
  \item Explain, based on the feasible region and your contours, whether the LP:
  \begin{itemize}
    \item is unbounded, 
    \item is infeasible, or
    \item has optimal solution(s); 
  \end{itemize}
\end{enumerate}
\end{enumerate}
\textbf{Stretch Questions}
\begin{enumerate}[resume]
\item The following parts involve \textbf{the same constraints} (and thus the same feasible region),
  but require you to 
  \begin{itemize}
    \item \textbf{find an appropriate (linear) objective function}\\
      \qquad{}or 
    \item \textbf{explain why it is impossible to find such a (linear) objective function.}
  \end{itemize}
  (Assume in each case that you are \emph{maximizing} your objective function.)

  \textbf{Hint:} Start with the geometry, then design contour lines, and then develop the function.
	\begin{enumerate}
	\item Find an objective function so that the LP has no optimal solution.
	\item Find an objective function so that the LP is unbounded.
	\item Find an objective function so that the LP has a unique optimal solution.
	\item Find an objective function so that the LP has exactly two optimal solutions.
	\item Find an objective function so that the LP has infinitely many optimal solutions.
	\item Find an objective function so that the LP is infeasible.
	\end{enumerate}
\end{enumerate}

\begin{figure*}[ht]
\begin{tikzpicture}[line cap=round,line join=round,>=triangle 45,x=0.30cm,y=0.30cm]
\foreach \x in {1,2,...,5}
\draw[shift={(10*\x,0)},color=black,line width=2pt,xstep=5] (0pt,3pt) -- (0pt,-3pt) node[below] {$\x$};
\foreach \x in {1,2,...,5}
\draw[shift={(0,10*\x)},color=black,line width=2pt,xstep=5] (3pt,0pt) -- (-3pt,0pt) node[left] {$\x$};
\draw [color=black!25, xstep=1,ystep=1,line width=0.25pt] (-5,-5) grid (55,55);
\draw [color=black!50, xstep=5,ystep=5,line width=0.5pt] (-5,-5) grid (55,55);
\draw[->,color=black, line width=2pt] (-5,0.0) -- (55,0.0);
\draw[->,color=black, line width=2pt] (0.0,-5) -- (0.0,55);
\draw (53,0.0) node[below right] {$x_{1}$};
\draw (0.0,53) node[above left ] {$x_{2}$};
\end{tikzpicture}
  \caption{Sketch and Shade the Feasible Region.}
  \tablabel{sketchFeasible}
\begin{equation*}\LP{2}{
    \text{maximize}&z&=&\multicolumn{3}{c}{f(x_{1},x_{2})}\\
  \text{subject to}& & & -x_1 & + &  x_2 & \leq & 2\\\notag
                   & & &  x_1 & - & 2x_2 & \leq & 1\\\notag
	           & & & \multicolumn{5}{c}{x_1,x_2\geq0}
	 }
\end{equation*}
\end{figure*}
%% End Exercise
\end{enumerate}
