\begin{enumerate}
\item
  Explain the Two-Phase Simplex Algorithm\footnote{You may choose to explain the ``Big M'' method if
  you prefer.}. In particular, 
\begin{itemize}
  \item    Explain the purpose of the Two Phase Simplex Algorithm,
  \item    Explain the implementation of the Two Phase Simplex Algorithm, and
  \item    Explain the various possible outcomes and their consequences.
\end{itemize}

\item
An LP has objective function \( \text{maximize } 3x_1 + 2 x_2 - x_3 \), slack
variables \(s_1\), \(s_2\), \(s_3\), and \(s_4\), four main constraints, and all variables
non-negative.  

\begin{enumerate}
  \item At any basic feasible solution, how many variables are basic and how many are non-basic?
  \item The current point is $(x_1,x_2,x_3,s_1,s_2,s_3,s_4)=(0,0,6,0,8,3,5)$. Is this a basic point
  or not, and how can you tell?

  \item You have the following options for moving from the current point, 
\begin{align*}
  \Vec{d}_1&=\begin{pmatrix*}[r]1& 0& 4& 0&           -3& 2& \phantom{-}1\end{pmatrix*}\\
  \Vec{d}_2&=\begin{pmatrix*}[r]0& 1& 1& 0& \phantom{-}1& 1& \phantom{-}1\end{pmatrix*}\\
  \Vec{d}_3&=\begin{pmatrix*}[r]0& 0& 2& 1&           -1& 1&           -2\end{pmatrix*}.
\end{align*}
\begin{enumerate}
	\item Which direction(s) are directions of improvement, and how can you tell?
	\item Explain how you can tell that the LP is unbounded.
\end{enumerate}
\end{enumerate}

\clearpage
\item Professor~Emmet (Doc) Brown is writing a grant proposal
that involves implementing a new algorithm he has developed.  He estimates that
it would require at least 1000~hours of programming by a professional consultant, but he can
substitute undergraduate programmers or graduate programmers.  The costs for an
hour of time for undergraduates, graduates, and consultants are, respectively,
\$4, \$10, and \$25.  

Doc's experience is that an hour of undergraduate time is approximately
equivalent to \nfrac{1}{5}~of an hour of professional time, and graduate student time is
roughly equivalent to \nfrac{3}{10}~hours of professional time.  

Programmers also require oversight.  Doc estimates that he will need to
invest \nfrac{1}{5}, \nfrac{1}{10}, and \nfrac{1}{20}~hour of his own time for each hour of programming
from undergrads, grads, and professionals;  he only has 164 hours to devote to
the programming aspects of the project, however.

Finally, due to the number of graduate students available, Doc can use at
most 500~hours of graduate student time.

\begin{enumerate}
\item Set up and solve a linear program for this situation.
\item Based on your final tableau, how much money could be saved if Doc
could find more of his own time to contribute to the project?
\end{enumerate}

\item
  Let \textbf{P} be the linear program
  \begin{equation*}
    \LP{2}{
    \text{maximize}  &z&= & 2x_1 &+& 3x_2 &      &    \\
    \text{subject to}& &  & 4x_1 &+& 3x_2 & \leq & 15 \\
                     & &  &  x_1 &+&  x_2 & \leq &  4 \\
		     & &  &\multicolumn{5}{c}{x_1, x_2 \geq 0}
			 }
  \end{equation*}

  Classify each of the following solutions as one of:
  \begin{itemize}
    \item infeasible solution,
    \item feasible boundary solution, or
    \item feasible interior solution.
  \end{itemize}

  \begin{multicols}{2}
    \begin{enumerate}
      \item $\Vec{x} = \begin{bmatrix}3\\3\end{bmatrix}$\label{33}
      \item $\Vec{x} = \begin{bmatrix}-1\\2\end{bmatrix}$\label{-12}
      \item $\Vec{x} = \begin{bmatrix}1\\1\end{bmatrix}$\label{11}
      \item $\Vec{x} = \begin{bmatrix}2\\2\end{bmatrix}$\label{22}
    \end{enumerate}
  \end{multicols}
\clearpage
\item
  Farmer John and Farmer Jane want to plant some or all of their 45~acres
  with wheat and/or corn.  They estimate that each acre of wheat will
  yield \$300~profit, while each acre of corn will yield \$200.  In their
  experience, labor (workers per acre) and fertilizer (tons per acre) required are as follows:\\[1ex]
  \begin{center}
    \begin{tabular}{lccc}
      \toprule
      & wheat	& corn	& available \\ \midrule
      labor	& 3	& 2	& 100 \\
      fertilizer& 2	& 4	& 120 \\
      \bottomrule
    \end{tabular}\\[1ex]
  \end{center}

  This yields the following linear program:
  \begin{equation*}
    \LP{2}{
    \text{maximize }\quad  & 200w &{}+{}& 300c &      &     \\
    \text{subject to }\quad&   3w &{}+{}&   2c & \leq & 100 \\
                           &   2w &{}+{}&   4c & \leq & 120 \\
                           &    w &{}+{}&    c & \leq &  45 \\
			   &   \multicolumn{5}{c}{c,w\ge 0}
			 }
   \end{equation*}

  Jane adds slack variables and uses the simplex algorithm to arrive at the following optimal
  tableau: (NOTE: You do not need to re-solve this if you do not wish to do so.  It is
  correct!)\\
  \begin{equation*}
    \begin{array}{r|rr|rrr|r}
	z &x_1&x_2& s_1 & s_2    &s_3&RHS \\ \hline
	1 & 0 & 0 &          25  & \nfrac{125}{2} & 0 & 10,000 \\ %\hline
	0 & 1 & 0 &\nfrac{ 1}{2} & \nfrac{-1}{4}  & 0 &     20 \\
	0 & 0 & 1 &\nfrac{-1}{4} & \nfrac{ 3}{8}  & 0 &     20 \\
	0 & 0 & 0 &\nfrac{-1}{4} & \nfrac{-1}{8}  & 1 &      5 
    \end{array}
  \end{equation*}

  Answer the following questions based on the above tableau:
  \begin{enumerate}
    \item If John and Jane have the opportunity to rent more land for
      planting at \$2000~per acre for the growing season, should they?
      Explain.\label{land}
    \item John and Jane find a fertilizer distributor who can sell them
      fertilizer at \$55~per ton.  Should they purchase more?\label{fertilizer}
    \item How much additional profit (if any) could John and Jane expect if
      they had an additional 10 workers?\label{labor}
    \item For parts~\ref{land} and~\ref{fertilizer}, what
      is the maximum John and Jane could rent/purchase and still have the same
      optimal basis?
    \item Look again at part~\ref{labor}.  If John and
      Jane have 10 extra workers, how much of their land should they plant
      with corn versus wheat?
  \end{enumerate}

\clearpage
\item
  Consider the sketch of the feasible region and solutions (points) shown below, where the arrows
  indicate the feasible side for each constraint:\\
 \begin{figure}[ht]
    \definecolor{gray25}{rgb}{0.25,0.25,0.25}
    \definecolor{gray33}{rgb}{0.33,0.33,0.33}
    \definecolor{gray75}{rgb}{0.75,0.75,0.75}
    \begin{tikzpicture}[line cap=round,line join=round,>=triangle 45,x=0.75cm,y=0.75cm]
    \draw [color=gray75,, xstep=1,ystep=1] (-1,-1) grid (20,16);
    \draw[->,color=black] (-1,0) -- (20,0);
    \foreach \x in {1,2,3,4,5,6,7,8,9,10,11,12,13,14,15,16,17,18,19}
      \draw[shift={(\x,0)},color=black] (0pt,2pt) -- (0pt,-2pt) node[below] {\scriptsize $\x$};
    \draw[color=black] (18.92,0.17) node [anchor=south west] { $x_1$};
    \draw[->,color=black] (0,-1) -- (0,16);
    \foreach \y in {1,2,3,4,5,6,7,8,9,10,11,12,13,14,15}
      \draw[shift={(0,\y)},color=black] (2pt,0pt) -- (-2pt,0pt) node[left] {\scriptsize $\y$};
    \draw[color=black] (0.18,15.17) node [anchor=west] { $x_2$};
    \draw[color=black] (0pt,-10pt) node[right] {\footnotesize $0$};
    \clip(-1,-1) rectangle (20,16);
    \draw[smooth,samples=100,domain=6.0:20.0] plot(\x,{2/3*((\x)-6)});
    \draw[smooth,samples=100,domain=11.0:20.0] plot(\x,{5/2*((\x)-11)});
    \draw[smooth,samples=100,domain=12.0:18.0] plot(\x,{(-3*((\x)-18))});
    \draw[smooth,samples=100,domain=0.0:8.0] plot(\x,{(2*(\x)+5)});
    \draw[smooth,samples=100,domain=0.0:20.0] plot(\x,{((-1)/8*((\x)-3)+16)});
    \begin{small}
    \fill [color=gray33] (0,0) circle (0.10);
    \draw[color=gray33] (0.44,0.53) node {$P_{0}$};
    \fill [color=gray33] (5,2) circle (0.1);
    \draw[color=gray33] (5.42,2.55) node {$P_{1}$};
    \fill [color=gray33] (6,0) circle (0.1);
    \draw[color=gray33] (6.00,0.53) node {$P_{2}$};
    \fill [color=gray25] (10,2.67) circle (0.1);
    \draw[color=gray25] (10.4,2.3) node {$P_{3}$};
    \fill [color=gray33] (8,4) circle (0.1);
    \draw[color=gray33] (8.42,4.53) node {$P_{4}$};
    \fill [color=gray25] (12.8181,4.5454) circle (0.1);
    \draw[color=gray25] (13.0,4.20) node {$P_{5}$};
    \fill [color=gray25] (15.82,6.55) circle (0.1);
    \draw[color=gray25] (16.32,6.55) node {$P_{6}$};
    \fill [color=gray33] (0,5) circle (0.1);
    \draw[color=gray33] (0.30,4.50) node {$P_{7}$};
    \fill [color=gray33] (1,5) circle (0.1);
    \draw[color=gray33] (1.42,4.69) node {$P_{8}$};
    \fill [color=gray33] (6,7) circle (0.1);
    \draw[color=gray33] (6.43,7.54) node {$P_{9}$};
    \fill [color=gray33] (11,8) circle (0.1);
    \draw[color=gray33] (11.59,8.53) node {$P_{10}$};
    \fill [color=gray25] (14.8181,9.5454) circle (0.1);
    \draw[color=gray25] (15.40,9.21) node {$P_{11}$};
    \fill [color=gray33] (2,9) circle (0.1);
    \draw[color=gray33] (2.57,9.00) node {$P_{12}$};
    \fill [color=gray33] (6,10) circle (0.1);
    \draw[color=gray33] (6.58,10.55) node {$P_{13}$};
    \fill [color=gray25] (4,13) circle (0.1);
    \draw[color=gray25] (4.59,12.74) node {$P_{14}$};
    \fill [color=gray33] (10,13) circle (0.1);
    \draw[color=gray33] (10.58,13.52) node {$P_{15}$};
    \fill [color=gray25] (5.35,15.71) circle (0.1);
    \draw[color=gray25] (5.89,15.38) node {$P_{16}$};
    \fill [color=gray33] (13,14) circle (0.1);
    \draw[color=gray33] (12.64,13.57) node {$P_{17}$};
    \fill [color=gray25] (13.09,14.74) circle (0.1);
    \draw[color=gray25] (13.65,15.26) node {$P_{18}$};
    \draw [->] (18,8) -- (17,9.5);
    \draw [->] (17,15) -- (15,16);
    \draw [->] (14,12) -- (12.5,11.5);
    \draw [->] (11,15) -- (10.875,14);
    \draw [->] ( 3,11) -- (3.80,10.60);
    \end{small}
    \end{tikzpicture}
   \caption{Feasible Region and Points}
   \figlabel{feasible}
 \end{figure}
 \begin{enumerate}
   \item Shade the feasible region and outline the boundary of the region.
   \item Which of the following successions of solutions could have arisen from:
   (1)~the simplex algorithm, (2)~an interior point algorithm, or
   (3)~neither?  Be sure to explain how you determined your answer.
   \begin{multicols}{2}
     \begin{enumerate}
       \item $P_{0}, P_{2}, P_{3}, P_{5}, P_{11}$ \label{P0P2P3P5P11}
       \item $P_{1}, P_{9}, P_{13}, P_{15}, P_{17}$ \label{P1P9P13P15P17}
       \item $P_{8}, P_{9}, P_{13}, P_{14}, P_{16}$ \label{P8P9P13P14P16}
       \item $P_{0}, P_{7}, P_{16}, P_{18}$ \label{P0P7P16P18}
       \item $P_{0}, P_{2}, P_{5}, P_{6}, P_{11}$ \label{P0P2P5P6P11}
     \end{enumerate}
   \end{multicols}
 \end{enumerate}
\end{enumerate}
\endinput
